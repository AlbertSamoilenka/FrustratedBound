\documentclass[12pt,a4paper]{article}
\usepackage[thinlines]{easytable} %for TAB useage
\usepackage{multirow}
\usepackage{booktabs}
\usepackage{array}
\usepackage{multicol}
\newcounter{fig}   \newcommand{\lbfig}[1]{\refstepcounter{fig}
\label{#1} }
\newcommand{\vphi}{\varphi}
\newcommand{\Tr}{{\rm Tr}}
\newcommand{\Atop}[2]{\genfrac{}{}{0pt}{}{#1}{#2}}
\newcommand{\half}{{\textstyle\frac{1}{2}}}
\newcommand{\threehalf}{{\textstyle\frac{3}{2}}}
\newcommand{\threefourth}{{\textstyle\frac{3}{4}}}
\newcommand{\fourth}{{\textstyle\frac{1}{4}}}
\newcommand{\bea}{\begin{eqnarray}}
\newcommand{\eea}{\end{eqnarray}}
\newcommand{\be}{\begin{equation}}
\newcommand{\ee}{\end{equation}}
\def\UDK#1{{\leftline{??? {#1}}}}
\def\diag{\mathop{{\rm diag}}\nolimits}
\def\Laa{\widetilde L}
\def\vecalpha{{\pmb{\alpha}}}
\def\vecgamma{{\pmb{\gamma}}}
\def\vecepsilon{{\pmb{\epsilon}}}
\def\vecvarepsilon{{\pmb{\varepsilon}}}
\def\vecpi{{\pmb{\pi}}}
\def\vecphi{{\pmb{\varphi}}}
\def\vecsigma{{\pmb{\sigma}}}
\def\vectau{{\pmb{\tau}}}
\def\vectheta{{\pmb{\theta}}}
\def\vecxi{{\pmb{\xi}}}
\def\vecomega{{\pmb{\omega}}}
\def\vecnabla{{\pmb{\nabla}}}
\def\bfph{{\pmb{\phi}}}
\newcommand{\da}{\dot\alpha}
\newcommand{\db}{\dot\beta}
\newcommand{\re}[1]{(\ref{#1})}
\newcommand{\BA}{\ensuremath{\mathbf{A}}}
\newcommand{\BB}{\ensuremath{\mathbf{B}}}
\newcommand{\Beta}{\ensuremath{\mathbf{\eta}}}
\newcommand{\BV}{\ensuremath{\mathbf{V}}}
\newcommand{\Bn}{\ensuremath{\mathbf{n}}}
\newcommand{\BW}{\ensuremath{\mathbf{W}}}
\newcommand{\BD}{\ensuremath{\mathbf{D}}}
\newcommand{\BP}{\ensuremath{\mathbf{P}}}
\newcommand{\BU}{\ensuremath{\mathbf{U}}}
\newcommand{\BH}{\ensuremath{\mathbf{H}}}
\newcommand{\BR}{\ensuremath{\mathbf{R}}}
\newcommand{\BJ}{\ensuremath{\mathbf{J}}}
\newcommand{\Balpha}{\ensuremath{\boldsymbol{\alpha}}}
\newcommand{\Bbeta}{\ensuremath{\boldsymbol{\beta}}}
\newcommand{\Bphi}{\ensuremath{\boldsymbol{\phi}}}
\newcommand{\Btau}{\ensuremath{\boldsymbol{\tau}}}
\newcommand{\Bpi}{\ensuremath{\boldsymbol{\pi}}}
\newcommand{\Bxi}{\ensuremath{\boldsymbol{\xi}}}
\newcommand{\Br}{\ensuremath{\mathbf{r}}}
\newcommand{\Bx}{\ensuremath{\mathbf{x}}}
\newcommand{\Bq}{\ensuremath{\mathbf{q}}}
\newcommand{\Bp}{\ensuremath{\mathbf{p}}}
\newcommand{\Bv}{\ensuremath{\mathbf{v}}}
\newcommand{\Bh}{\ensuremath{\mathbf{h}}}
\newcommand{\Ba}{\ensuremath{\boldsymbol{a}}}
\newcommand{\BPhi}{\ensuremath{\boldsymbol{\Phi}}}
\newcommand{\Bell}{\ensuremath{\boldsymbol{\ell}}}
\newcommand{\Bsigma}{\ensuremath{\boldsymbol{\sigma}}}

%other letters &operators
\newcommand{\Z}{\ensuremath{\mathds{Z}}}
\newcommand{\NN}{\ensuremath{\mathds{N}}}
\newcommand{\R}{\ensuremath{\mathds{R}}}
\newcommand{\SU}{SU(2)}
\newcommand{\su}{{su(2)}}
\newcommand{\e}{\mathrm{e}}
\newcommand{\rt}{\tilde{r}}
\newcommand{\di}{\mathrm{d}}
\newcommand{\dd}{\mathrm{d}}
\newcommand{\ii}{\mathrm{i}}
\newcommand{\CK}{\ensuremath{\mathcal{K}}}
\newcommand{\CH}{\ensuremath{\mathcal{H}}}
\newcommand{\CL}{\ensuremath{\mathcal{L}}}
\newcommand{\CB}{\ensuremath{\mathcal{B}}}
\newcommand{\CM}{\ensuremath{\mathcal{M}}}
\newcommand{\CG}{\ensuremath{\mathcal{G}}}
\newcommand{\CE}{\ensuremath{\mathcal{E}}}
\newcommand{\CA}{\ensuremath{\mathcal{A}}}
\newcommand{\CF}{\ensuremath{\mathcal{F}}}
\newcommand{\tr}{\mbox{tr}}
\newcommand{\la}{\lambda}
\newcommand{\ta}{\theta}
\newcommand{\vf}{\varphi}
\newcommand{\ka}{\kappa}
\newcommand{\al}{\alpha}
\newcommand{\ga}{\gamma}
\newcommand{\de}{\delta}
\newcommand{\si}{\sigma}
\newcommand{\bomega}{\mbox{\boldmath $\omega$}}
\newcommand{\bsi}{\mbox{\boldmath $\sigma$}}
\newcommand{\bchi}{\mbox{\boldmath $\chi$}}
\newcommand{\bal}{\mbox{\boldmath $\alpha$}}
\newcommand{\bpsi}{\mbox{\boldmath $\psi$}}
\newcommand{\brho}{\mbox{\boldmath $\varrho$}}
\newcommand{\beps}{\mbox{\boldmath $\varepsilon$}}
\newcommand{\bxi}{\mbox{\boldmath $\xi$}}
\newcommand{\bbeta}{\mbox{\boldmath $\beta$}}
\newcommand{\pa}{\partial}
\newcommand{\Om}{\Omega}
\newcommand{\vep}{\varepsilon}

\usepackage{ulem}
%----------Input encoding---------------
\usepackage[utf8]{inputenc}
%----------Font encoding----------------
\RequirePackage[T2A]{fontenc}
%----------Babel------------------------
\RequirePackage[english]{babel}
%----------Indent for a new section-----
\RequirePackage{indentfirst}
%----------AMS formulae-----------------
\RequirePackage{amsmath,amssymb,amsfonts}
%----------Section's title appearence---
\RequirePackage{titlesec}
%----------Table of Contents Design-----
\RequirePackage[titles]{tocloft}
%----------Including pictures-----------
\RequirePackage{graphicx}
%----------[H] Option for PUT IT HERE!--
\RequirePackage{float}
%----------Style of float's captions----
\RequirePackage{caption}
%----------Inparagraph floats-----------
\RequirePackage{wrapfig}
%----------Hyperref---------------------
\RequirePackage[colorlinks=true,allcolors=blue,
unicode=true,hypertexnames=false]{hyperref}
\hypersetup{pdfstartview={XYZ null null 1.25}}

\usepackage[left=2cm,right=2cm,top=2cm,bottom=2cm]{geometry}




\graphicspath{{images/}}


\title{Frustrated Bound}
\author{
    {\large A.~Samoilenka}$^{\dagger}$
    and {\large Ya. Shnir}$^{\dagger \star}$
    \\ \\
    \\ $^{\dagger}${\small Department of Theoretical Physics and Astrophysics}\\
    {\small Belarusian State University, Minsk 220004, Belarus}
    \\ $^{\star}${\small BLTP, JINR, Dubna, Russia}
}


\begin{document}
\maketitle

\begin{abstract}

\end{abstract}

\section{Frustrated magnets model}
We have energy density \cite{Sutcliffe:2017aro}:
\be\label{energy}
\CE=-\frac{I_1}{2}(\pa_i m_j)^2 + \frac{I_2}{2}(\Delta m_i)^2 + H (1-m_3)
\ee 

Where $i,j=1,2,3$ and $m_i m_i=1$. Recall that hence 
\be\label{ort}
\pa_i m_j\pa_k m_j=-m_j\pa_i\pa_k m_j
\ee

Lets introduce matrix
\be\label{matrix}
M_{ij}=m_i\Delta m_j
\ee

Thus for example we get $\pa_i m_j\pa_i m_j=-m_j\pa_i^2 m_j=-Tr M$. In the same manner we can rewrite energy as:
\be
\CE=\frac{I_1}{2}Tr M + \frac{I_2}{2}Tr(M M^T) + H V(m_3)
\ee

Let's do now little trick which will help us get better bound later. If $m_i$ minimizes functional \re{energy}, then it should be minimal also relative scaling $x\rightarrow\lambda x$:\footnote{Here and farther we drop the integrals just for the sake of brevity}
\be
\left. \frac{d\CE}{d\lambda}\right|_{\lambda=1}= \frac{I_1}{2}Tr M - \frac{I_2}{2}Tr(M M^T) + 3 H V(m_3) = 0
\ee

Hence we can rewrite energy introducing arbitrary parameter $\alpha$:
\be\label{trick}
\CE= \CE + \alpha\left(\left. \frac{d\CE}{d\lambda}\right|_{\lambda=1}\right)= \frac{I_1}{2}(1+\alpha)Tr M + \frac{I_2}{2}(1-\alpha)Tr(M M^T) + (1+3\alpha)H V(m_3)
\ee

Actually this trick shoves the way to potentially improve any topological bound: if one thinks that formula for bound is not the best he can simply redefine parameters in the formula by this procedure and maximize it with respect to $\alpha$. 

\section{Usual Faddeev-Skyrme model}
Now lets consider usual Faddeev-Skyrme model with Skyrme term quartic in derivatives and positive parameter for second in derivatives term. Following notation of \cite{harland}:\footnote{we keep the notation for the field $m_i$ instead of usual symbol $\vec \phi$}
\be
\CE_{FS}=\alpha_2 (\pa_i m_j)^2 + \frac{\alpha_4}{2}\left[ (\pa_i m_j \pa_i m_j)^2 - \pa_i m_k \pa_j m_k \pa_j m_n \pa_i m_n \right]+\alpha_0 V
\ee

Then using \re{ort} and \re{matrix} we obtain the next form for $\CE_{FS}$
\be
\CE_{FS}=-\alpha_2 Tr M + \frac{\alpha_4}{2}\left[ (Tr M)^2 - (m_k \pa_i \pa_j m_k)^2 \right]+\alpha_0 V
\ee

As $(m_k \pa_i \pa_j m_k)^2\geq0$ we get:
\be
\CE_{FS}\leq -\alpha_2 Tr M + \frac{\alpha_4}{2}(Tr M)^2 +\alpha_0 V
\ee

Cauchy-Schwartz inequality states that $(Tr M)^2\leq d\  Tr M^2$, where $d=dim(M)=3$ -- dimension of the matrix. Thus we get:
\be
\CE_{FS}\leq -\alpha_2 Tr M + \frac{3\alpha_4}{2}Tr M^2 +\alpha_0 V
\ee

Next we can recall that any matrix could be represented as $M=M_s+M_a$, where $M_s^T=M_s$ is a symmetric part and $M_a^T=-M_a$ antisymmetric. And $Tr M_a=Tr(M_s M_a)=0$ because $Tr(M_s M_a)=(M_s)_{ij}(M_a)_{ji}=-(M_s)_{ji}(M_a)_{ji}$. Then we get
$$
TrM^2=Tr M_s^2 + Tr M_a^2
$$
\be
Tr(M M^T)=Tr M_s^2 - Tr M_a^2
\ee

Where $Tr M_a^2= (M_a)_{ij}(M_a)_{ji}=-(M_a)_{ij}(M_a)_{ij}\leq 0$, hence $Tr M^2 \leq Tr(M M^T)$ and we have new upper bound for $\CE_{FS}$:
\be\label{FS}
\CE_{FS}\leq -\alpha_2 Tr M + \frac{3\alpha_4}{2}Tr (M M^T) +\alpha_0 V
\ee

\section{Bound for frustrated magnets}
Now comparing \re{trick} and \re{FS} we see that in order for RHS to be equal we have to set:
$$
\alpha_2=-\frac{I_1}{2}(1+\alpha)
$$
\be\label{param}
\alpha_4=\frac{I_2}{3}(1-\alpha)\geq0
\ee
$$
\alpha_0=(1+3\alpha)H
$$

Thus we obtained the bound for the Frustrated magnets model with respect to Faddeev-Skyrme model, but with $I_1<0$, and hence positive sign for first term:
\be\label{geqFS}
\CE\geq\CE_{FS}
\ee

and corresponding redefinition of the parameters \re{param}. Though we still have free parameter $\alpha$, lets try to fix it. Obviously we should pick $\alpha$ so that $\CE_{FS}$ would be maximal for the given set of parameters in the model \re{energy}.

Now lets use general formula for the topological bound by Harland \cite{harland}:
\be
\CE\geq3^{3/8}16\pi^2\sqrt{\alpha_2\alpha_4}Q^{3/4}\left( 1+\frac{1}{3}\frac{k}{1+\sqrt{1+k}} \right)\sqrt{\frac{2}{1+\sqrt{1+k}}}
\ee

where 
\be
k=\left( \frac{2^7 3^{11/4}}{7^3 \pi^{3/2}} \right)^2 \frac{\alpha_4\alpha_0}{\alpha_2^2}
\ee

\section{Generalization of scaling theorem}
We can redefine so that 
\be\label{energyre}
\CE=-(\pa_i m_j)^2 + (\Delta m_i)^2 + H V
\ee 

Let's consider more general changes for $x$, in form
\be
x^i\rightarrow f^i(x^j)
\ee

so that now it has non equal diagonal elements, non-diagonal elements and depends on $x^i$, but still close to $x$, with small $\lambda_{ij}$:
\be
\frac{\pa x^j}{\pa f^i}=\delta_{ij}+\lambda_{ij}
\ee

then $d^3 f=(1-Tr\lambda)d^3 x$ and $\frac{\pa}{\pa f^i}=\pa_i+\lambda_{ij}\pa_j$. For solution first variation of energy is zero hence:
\be
\int d^3 x (-2\pa_i\vec{m}\pa_j\vec{m}+4\pa_i\pa_j\vec{m}\Delta\vec{m}+2\pa_j\vec{m}\Delta\vec{m}\pa_i-\delta_{ij}\CE)\lambda_{ij}=0
\ee

then
\be
-2\pa_i\vec{m}\pa_j\vec{m}+4\pa_i\pa_j\vec{m}\Delta\vec{m}-2\pa_i(\pa_j\vec{m}\Delta\vec{m})-\delta_{ij}\CE=0
\ee

in particular we get
\be
\pa_i\vec{m}^2+\Delta\vec{m}^2-3 H V=2\pa_i(\pa_i\vec{m}\Delta\vec{m})
\ee

\section{Negative energy density}
Let's consider $m_1=f(r), m_2=0$, then
\be
\CE\sim-f'^2+(f''+2\frac{f'}{r})^2+\frac{H}{2}f^2
\ee

for small $f$. In order for second to be zero we obtain $f'r^2=c_1$, and hence $f=-\frac{c_1}{r}+c_2$. let's set $c_2=0$. then $\CE\sim-\frac{c_1^2}{r^4}+\frac{H}{2}\frac{c_1^2}{r^2}$. Hence for small $r$ energy density will be negative!

\section{Conclusions}

%The main useful feature that could be obtained from this paper is that: it is profitable some times to consider $M_{ij}=m_i\Delta m_j$ matrix instead of $D_{ij}=\pa_i m_k \pa_j m_k$ (notice the summation goes for different induces). And the trick \re{trick} could be be useful for different models.

\section{To do}
-Can I prove that energy can be negative globally, not only locally?

-Can I make Wick rotation and analitically continue solution for $I_1>0$?

\section*{Acknowledgements}
A.S. is very sorry to professors R. Poghossian and A. Belavin who's lectures during SymPhys XIV he entirely wasted on the derivation of this bound.

\begin{small}
\begin{thebibliography}{13}
%%%%%%%%%%%%%%%%%%%%%%%%%%%%%%%%%%%%%%%%%%%%%%%%%%%%%%%%%%%%%%%%%%%%%%%%%%%%%%
\bibitem{Sutcliffe:2017aro}P.~Sutcliffe, Skyrmion knots in frustrated magnets, 
Phys.\ Rev.\ Lett.\  {\bf 118} (2017) no.24,  247203
%%%%%%%%%%%%%%%%%%%%%%%%%%%%%%%%%%%%%%%%%%%%%%%%%%%%%%%%%%%%%%%%%%%%%%%%%%%%%%
\bibitem{harland}Harland, D. (2014). Topological energy bounds for the Skyrme and Faddeev models with massive pions. Physics Letters B, 728, 518-523.

\end{thebibliography}
\end{small}

%\bibliographystyle{gost7103u-m_non-bold}
%\bibliography{bib}


\end{document}
